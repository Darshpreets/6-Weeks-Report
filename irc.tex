
\subsection{Internet Relay Chat}
Internet Relay Chat (IRC) is an application layer protocol that facilitates transfer of messages in the form of text. The chat process works on a client/server model of networking. IRC clients are computer programs that a user can install on their system. These clients are able to communicate with chat servers to transfer messages to other clients. It is mainly designed for group communication in discussion forums, called channels, but also allows one-to-one communication via private message as well as chat and data transfer, including file sharing.
Client software is available for every major operating system that supports Internet access.IRC is an open protocol that uses TCP and, optionally, TLS. An IRC server can connect to other IRC servers to expand the IRC network. Users access IRC networks by connecting a client to a server. There are many client implementations, such as mIRC, HexChat and irssi, and server implementations, e.g. the original IRCd. Most IRC servers do not require users to register an account but a user will have to set a nickname before being connected.

IRC has a line-based structure with the client sending single-line messages to the server,receiving replies to those messages and receiving copies of some messages sent by other clients. In most clients, users can enter commands by prefixing them with a '/'. Depending on the command, these may either be handled entirely by the client, or  passed directly to the server, possibly with some modification.

The basic means of communicating to a group of users in an established IRC session is through a channel. Channels on a network can be displayed using the IRC command LIST, which lists all currently available channels that do not have the modes +s or +p set, on that particular network.
Users can join a channel using the JOIN command, in most clients available as /join \#channelname. Messages sent to the joined channels are then relayed to all other users.


